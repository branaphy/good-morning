% This is samplepaper.tex, a sample chapter demonstrating the
% LLNCS macro package for Springer Computer Science proceedings;
% Version 2.20 of 2017/10/04
%
\documentclass[runningheads]{llncs}
%
\usepackage{lipsum}  
\usepackage{graphicx}
% Used for displaying a sample figure. If possible, figure files should
% be included in EPS format.
%
% If you use the hyperref package, please uncomment the following line
% to display URLs in blue roman font according to Springer's eBook style:
% \renewcommand\UrlFont{\color{blue}\rmfamily}
% The following line is for merging abstracts into proceedings later:
% \documentclass[abstract-main.tex]{subfiles} 
\begin{document}
%
\title{Morning Mist vs. Evening Clarity: A Comparative Study of Vowel Space Area and Speech Rate}
%
\titlerunning{A Comparative Study of Vowel Space Area and Speech Rate}
% If the paper title is too long for the running head, you can set
% an abbreviated paper title here
%
\author{Brandi Hongell\inst{1} \and
Yi Lei\inst{1} \and
Xueying Liu\inst{1} \and 
Page Ouyang\inst{1} \and
Yilan Wei\inst{1}}
%
\authorrunning{B. Hongell, Y. Lei, X. Liu, P. Ouyang, Y. Wei}
% First names are abbreviated in the running head.
% If there are more than two authors, 'et al.' is used.
%
\institute{University of Groningen, Campus Frysl\^an}
%
\maketitle              % typeset the header of the contribution
%
%
%
\section*{Abstract}
\subsection*{Background}
\indent Fatigue and its impact on cognitive functions have been subjects of extensive research over the years. The state of tiredness, 
often influenced by the circadian rhythm, can significantly affect an individual’s motor and cognitive abilities 
\cite{zhang2015influence}, thereby potentially altering the acoustic properties of speech.

The motivation behind this study stems from the curiosity to explore how "morning grogginess" may impact speech production, which could 
have substantial implications in fields like voice recognition technology, speech pathology, and human-computer interaction. The insights 
garnered from this study could significantly contribute to real-world applications, notably in the realm of fatigue/tiredness 
detection technologies. Understanding how speech parameters may be altered by tiredness creates foundational knowledge for developing 
fatigue/tiredness detection systems, particularly for drivers and machine operators. Such advancements could be pivotal in enhancing road 
safety by alerting drivers of fatigue in real-time, thereby reducing the likelihood of fatigue-related accidents. 

Additionally, the findings could fuel the creation of more adaptive voice recognition systems that account for variations in 
speech due to differing levels of alertness, such as smart home systems. In a time where voice-activated systems are becoming increasingly 
common in our daily lives, tailoring these technologies to accurately respond to speech, regardless of the user's state of tiredness, is 
inevitable and imperative.

% the environments 'definition', 'lemma', 'proposition', 'corollary',
% 'remark', and 'example' are defined in the LLNCS documentclass as well.
%

\subsection*{Methods}
Each of the authors recorded themselves speaking the same phrase, twice per day, for three consecutive days. The recordings were made on 
each respective author's cell phone. The first recording was made immediately upon waking, and the second was made at 20:00. The time 
immediately after waking was chosen as a reliable control for tired speech, and 16:00 was chosen as the reliable control for alert speech.

The control phrase was: "\textbf{Ah,} good morning! I'm going to \textbf{brew} a cup of \textbf{coffee}." This phrase was chosen for
several reasons:
\begin{itemize}
    \item The phrase offers a relation to the abstract topic.
    \item The phrase offers multiple vowel sounds.
    \item The length of the phrase is brief, yet long enough to accurately measure the speech rate.
\end{itemize}

After collecting the recordings, the data was measured to analyze patterns in two specific areas:
\begin{itemize}
    \item Vowel space area in vowels a, i, and  u.
    \item Overall speech rate.
\end{itemize}

The measurement of vowel space area was computed with an R script \cite{Rscript}, and the measurement of speech rate was calculated using 
a PRAAT script \cite{praatscript}. 
\subsection*{Results}

Here comes the results part.

\subsection*{Conclusion}
Here comes the results part.

\keywords{Tiredness  \and Fatigue \and Speech Rate  \and Vowel Space Area}

\subsection*{Authors' contributions}
Each author participated in data collection and methodology. Xueying compiled the final plots using each author's data. Page and Xueying
prepared the presentation. Yilan, Brandi, and Yi presented the abstract. Yilan assisted in writing the methodology and introduction. Each
author contributed to the conclusion. Brandi prepared the GitHub repository and compiled the LaTeX document.
%
% ---- Bibliography ----
%
% BibTeX users should specify bibliography style 'splncs04'.
% References will then be sorted and formatted in the correct style.
%
\bibliographystyle{refs-style}
\bibliography{refs}

%

\end{document}

